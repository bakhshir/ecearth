


\appendix
\chapter{Appendix}
\section{Compilation}

\begin{lstlisting}[caption={Compilation functions},label={app:comp},language=Bash]
# =========== COMPILATIONS ==================
# config for compilations
cconf32()
{
    cd ${ece32b}sources
    ec-conf -p neuron-intel-intelmpi config-build.xml
    cd -
}
#oasis (add realclean to the make command)
coas32()
{
    cd ${ece32b}sources/oasis3-mct/util/make_dir
    make BUILD_ARCH=ecconf -f TopMakefileOasis3
    cd -
}
# xios. "--full" means recompile from scratch.
cxios32()
{
    cd ${ece32b}sources/xios-1.0
    ./make_xios --arch ecconf --use_oasis oasis3_mct --netcdf_lib netcdf4_seq --full --job 8
    cd -
}
# runoff-mapper (make clean)
crunoff32()
{
    cd ${ece32b}sources/runoff-mapper/src
    make
    cd -
}
# ifs (make realclean)
cifs32()
{
    cwd=$PWD 
    cd ${ece32b}sources/ifs-36r4
    make BUILD_ARCH=ecconf -j 10 lib
    if [ $? == 0 ]; then make BUILD_ARCH=ecconf master; fi
    cd $cwd
}

# nemo (./makenemo clean)
# requires arg. Can be ORCA025L75_LIM3, ORCA025L75_LIM3_standalone, 
                    ORCA1L75_LIM3, ORCA1L75_LIM3_standalone,
cnemo32()
{
    cd ${ece32b}sources/nemo-3.6/CONFIG
    ./makenemo -n $@ -m ecconf -j 8
    cd -
} 

# tm5 (pass -n to cleanup before)
ectm532()
{
    cd ${ece32b}sources/tm5mp
    setup_tm5 $@ ecconfig-ecearth3.rc
    cd -
}
\end{lstlisting}

\begin{lstlisting}[caption={Run functions},label={app:run},language=Bash]
# run IFS+NEMO
function sub32 {
    # run COUPLED
    local hires=''
    (( $# )) && [[ $1 = '-hr' ]] && hires=-hires
    
    cd ${ece32b}runtime/classic
    ec-conf -p neuron config-run${hires}.xml
    expn=$(get_ece_config_param.py config-run${hires}.xml GENERAL EXP_NAME)  ||  return 1
    nifs=$(get_ece_config_param.py config-run${hires}.xml IFS NUMPROC)    ||  return 1
    nnem=$(get_ece_config_param.py config-run${hires}.xml NEM NUMPROC)    ||  return 1
    nxio=$(get_ece_config_param.py config-run${hires}.xml XIO NUMPROC)    ||  return 1
    ppno=$(get_ece_config_param.py config-run${hires}.xml neuron PROC_PER_NODE) ||  return 1 
    totp=$((nifs+nnem+nxio+1))
    node=$(( totp/ppno ))
    (( totp % ppno )) && (( node+=1 ))
    echo "Submit: ifs${hires}(${nifs}) + nemo${hires}(${nnem}), exp=${expn} [-N${node} -n${totp} --ntasks-per-node=${ppno}]"
    sbatch -N${node} -n${totp} --exclusive --account=ecearth --ntasks-per-node=${ppno} -o "out/${expn}${hires}.out.001" -e "out/${expn}${hires}.err.001" ./ece-ifs+nemo.sh
    cd -
}

function sub32atm {
    # run IFS alone
    local hires=''
    (( $# )) && [[ $1 = '-hr' ]] && hires=-hires

    cd ${ece32b}runtime/classic
    ec-conf -p neuron config-run${hires}.xml
    expn=$(get_ece_config_param.py config-run${hires}.xml GENERAL EXP_NAME)  ||  return 1
    nifs=$(get_ece_config_param.py config-run${hires}.xml IFS NUMPROC)    ||  return 1
    ppno=$(get_ece_config_param.py config-run${hires}.xml neuron PROC_PER_NODE) ||  return 1
    node=$(( nifs/ppno ))
    (( nifs % ppno )) && (( node+=1 ))
    echo "Submit: ifs${hires}(${nifs}), exp=${expn} [-N${node} -n${nifs} --ntasks-per-node=${ppno}]"
    sbatch -N${node} -n${nifs} --exclusive --account=ecearth --ntasks-per-node=${ppno} -o "out/${expn}${hires}.out.001" -e "out/${expn}${hires}.err.001" ./ece-ifs.sh
    cd -
}

function sub32oce {
    # run NEMO alone
    local hires=''
    (( $# )) && [[ $1 = '-hr' ]] && hires=-hires

    cd ${ece32b}runtime/classic
    ec-conf -p neuron config-run${hires}.xml
    expn=$(get_ece_config_param.py config-run${hires}.xml GENERAL EXP_NAME)  ||  return 1
    nnem=$(get_ece_config_param.py config-run${hires}.xml NEM NUMPROC)    ||  return 1
    nxio=$(get_ece_config_param.py config-run${hires}.xml XIO NUMPROC)    ||  return 1
    ppno=$(get_ece_config_param.py config-run${hires}.xml neuron PROC_PER_NODE) ||  return 1
    totp=$((nnem+nxio))
    node=$(( totp/ppno ))
    (( totp % ppno )) && (( node+=1 ))
    echo "Submit: nemo${hires}(${nnem}), exp=${expn} [-N${node} -n${totp} --ntasks-per-node=${ppno}]"
    sbatch -N${node} -n${totp} --exclusive --account=ecearth --ntasks-per-node=${ppno} -o "out/${expn}${hires}.out.001" -e "out/${expn}${hires}.err.001" ./ece-nemo.sh
    cd -
}
\end{lstlisting}



% \section{Content of MY\_SRC directory}
% 
% (Future) Routines adapted for running ORCA025 (located in MY\_SRC directory)
% \begin{itemize}
%   \item closea.F90 : Caspian+Azov+Great Lakes+Lake Victoria
%   \item dommsk.F90 : read a shlat2d.nc file (no strait hard coded).
%   \item stpctl.F90 : test to detect NaN for velocities (to force NEMO to stop)
%   \item trabbl.F90 : flag to use k-level instead of depth for bbl threshold (Drakkar)
%   \item trazdf.F90 : fix on negative salinity removed (Drakkar)
%   \item zdftke.F90 : decrease kz background under sea ice + no Langmuir under sea ice (Drakkar)
% \end{itemize}
% 
% For now, a few new inputs in namelist\_cfg are required to run this ORCA025\_LIM3 configuration:
% \begin{itemize}
%   \item nambbl: ln\_kriteria (T/F).
%   \item namzdf\_tke: rn\_ebbice, nn\_havti (0/1), ln\_nolcice (T/F).
% \end{itemize}
% 
% 


\section{Creating OASIS restart files}
% from wiki: https://dev.ec-earth.org/projects/ecearth3/wiki/Creating_new_grid_files_for_Oasis

In order to couple IFS and Nemo, Oasis needs a number of files describing their respective grids. The files needed are

\begin{itemize}
  \item \code{grids.nc} - two identical atmospheric grids named A\#\#\# and L\#\#\# with coordinates and grid cell corners. The L-grid has a mask in masks.nc which includes lakes. One grid with coordinates for the oceanographic grid.
  \item \code{areas.nc} - two identical atmospheric grids with cell areas for the A\#\#\#-grid and L\#\#\#-grid described above. One grid with cell areas for the oceanographic grid.
  \item \code{mask.nc} - two different atmospheric masks, one with lakes and one without lakes. One oceanographic mask.
  \item \code{runoff\_maps.txt} - run-off maps for IFS
  \item Oasis restart files
  \item \code{rmp\_*.nc} - weight files used in the interpolation
\end{itemize}


This page details a procedure to prepare the above files for EC-Earth 3. The example below prepares the coupling between the horizontal resolutions T511 (IFS) and ORCA025 (Nemo) but is easily adapted to another configuration. Note that although the attached scripts are written in Matlab the operations involved are quite straightforward and can probably be implemented with the user's preferred tool.

Prerequisites are,
\begin{itemize}
  \item IFS:
        an initial file for surface fields on the model Gaussian grid with the correct horizontal resolution (i.e., ICMGG????INIT)
        runoff-maps from the IFS configuration T159 + landsea mask for T159 (both available in the default EC-Earth 3 setup)
  \item Nemo:
        an existing setup with the correct horizontal resolution (available in the default EC-Earth 3 setup)
        a Nemo mesh file with the correct horizontal resolution (can be generated by running Nemo with namelist option \code{'nn\_msh=1'})        
\end{itemize}


To get an IFS initial file for T511, run PrepIFS and download the ICMGG????INIT file. See the documentation at ECMWF for further details on how to run PrepIFS.
First, create grids.nc

Extract information about the number of longitude grid cells at each latitude with cdo,

\begin{verbatim}
 cdo griddes t511/ICMGG????INIT > rowlon.txt 
\end{verbatim}



Extract information about grid latitudes,

\begin{verbatim}
cdo -R griddes t511/ICMGG????INIT > yvals.txt  
\end{verbatim}


In the attached matlab script creategrid.m, update the following variables according to,

    'grids\_ifile' - path to an existing grids.nc file with the correct Nemo grid resolution
    'grids\_ofile' - the output grid.nc filename, e.g., t511/grids.nc
    'nemo\_grid' - the oasis name for the Nemo grid (e.g., Ot25)
    'ifs\_grid' - the oasis name for the IFS grid (e.g., A256)
    'rowlon' - the data from the rowlon.txt file above
    'yvals' - the data from the yvals.txt file above

Running the script will now produce the grids.nc-file.

Creating the areas.nc file is similar to the above procedure. Use the attached matlab script createareas.m and follow the above (grids.nc) instructions.

Next step, create \code{runoff\_maps.txt} Runoff maps are interpolated from the T159 setup with the help of the T159 and the target resolution landsea mask. For T511 we start by using cdo to extract and convert the T511 lsm to netcdf,

\begin{verbatim}
cdo selcode,172 t511/ICMGG????INIT t511/lsm.grb   # 172=landseamask
cdo -f nc -t ecmwf setgrid,g348528x1 t511/lsm.grb t511/lsm.nc   # 348528=grid size, use cdo info to find out  
\end{verbatim}


\iffalse

The actual runoff\_maps.txt file is created in two steps, the second step requiring some manual input. In the script createromap.m, update the following variables at the top of the script,

    't159_runoff_path' - path to existing T159-runoff_maps.txt file
    't159_grids_path' - path to existing T159 grids.nc file
    't159_lsm_path' - path to existing T159 landsea mask
    'target_grids_path' - path to existing target configuration grids.nc file, e.g., t511/grids.nc
    'target_lsm_path' - path to existing target configuration landsea mask, e.g., t511/lsm.nc
    'target_oasname' - Target configuration name in Oasis, e.g., for T511 it is A256
    'target_runoff_path' - output path for target configuration runoff_maps.txt file, e.g. t511/runoff_maps.txt

Running the script will create an intermediate runoff_maps.txt-file. The intermediate file do not describe lakes correctly and another script, correct_runoffmap_t511.m, is required. Prior to running this second script it needs to be updated such that all lakes are explicitly masked in the script. This is a manual task which can be simplified by plotting the intermediate runoff_maps.txt to locate the lakes. One way of plotting the runoff-file is by converting it to KML-format with the matlab script plot_runoff_maps.m (which in turn uses river2kml.awk and ocean2kml.awk). The KML-file can then be imported and plotted in, e.g., Google Earth. Notice that plot_runoff_maps.m needs some tweaking depending on what you would like to plot (inactive points=-2, lakes=-1, active points>0). For the current high resolution setups (T511 and T799) a drainage basin for the Aral Sea is explicitly added in the correct_runoff-scripts, this drainage basin is defined as a polygon in the file aralSeaDrainageBasin_lonLatPolygon.txt.
For reference, the correction of the runoff maps for T799 and T255 is also given here,

    correct_runoffmap_t799.m
    correct_runoffmap_t255.m (without the Aral Sea drainage basin)
\fi

NB: createromap.m depends on the Matlab Map toolbox routine distance(...), hence if the Map toolbox is not available on your system the createromap.m will fail. In such case the distance(...)-function can be replaced with a similar (but free) function available at the Matlab Central

Create mask.nc

Run Nemo with the namelist option \code{'nn\_msh=1'} to generate a mesh file with the correct horizontal resolution. Next, update the following variables in the attached createmask.m file,

\begin{itemize}
  \item 'mesh\_file' - Path to Nemo mesh file
  \item 'ifs\_lsm\_file' - Path to IFS land sea mask
  \item runoff\_maps\_file' - Path to the 'runoff\_maps.txt'-file
  \item 'masks\_ofile' - Path to the output mask file
  \item 'nemo.oasname' - Nemo configuration name in Oasis, e.g., 'Ot25'
  \item 'ifs.size' - , e.g., 348528 for T511
  \item 'ifs.oasname' -, e.g., 'A256' 
\end{itemize}

Running the script will create a new 'mask.nc' for the given resolution.

Create Oasis restart files:
The variables in an Oasis restart file is either a flux variable or a state variable. Currently (Dec. 2012) IFS sends only flux variables while Nemo sends only state variables. The fluxes can be initialized to zero by using cdo to create an array of the appropriate size. For example, the T511 field TAUY\_OCE can be created as

\begin{verbatim}
cdo -f nc -b 64 setname,TAUY_OCE -const,0,g348528x1 TAUY_OCE
\end{verbatim}


State variables are an estimate of an initial state and should not be set to zero. Use either an existing restart file with the correct horizontal resolution as an approximation to the initial state or create a new one using a NEMOVAR analysis. See issue \#71 for a discussion on possible improvements/problems to this "restart file creation"-approach.


Create Oasis weight files, rmp\_*.nc:

When missing, the weight files are created on the fly in a regular EC-Earth 3 run. Set oas\_numproc=1 in your run script and start EC-Earth at the required resolution.

\begin{lstlisting}[caption={Compilation functions},label={app:comp},language=Bash]
#!/usr/bin/env bash
# dir=~/DATA/ecearth-3/oasis/T511-ORCA025
#d1=${dir}/rst                  # from SPECS
#d2=${dir}/rst_original         # from distro 3.1

for f in O*
do
    echo $f
    case $(basename $f) in
        OIceFrac)
            cdo chname,OIceFrac,OIceFrc $f $f_new
            ;;
        O_IceTck)
            cdo chname,O_IceTck,OIceTck $f $f_new
            ;;
        O_SnwTck)
            cdo chname,O_SnwTck,OSnwTck $f $f_new
            ;;
        *)
            continue
    esac
done

cdo merge O_AlbIce OIceFrac_new O_IceTck_new O_SnwTck_new O_SSTSST
O_TepIce rstos_original.nc


cdo chname,CALV_OCE,A_Calving       CALV_OCE CALV_OCE_new
cdo chname,DQDT_ICE,A_dQns_dT       DQDT_ICE DQDT_ICE_new
cdo chname,EVAP_ICE,A_Evap_ice      EVAP_ICE EVAP_ICE_new
cdo chname,EVAP_TOT,A_Evap_total    EVAP_TOT EVAP_TOT_new
cdo chname,PRCP_LIQ,A_Precip_liquid PRCP_LIQ PRCP_LIQ_new
cdo chname,PRCP_SOL,A_Precip_solid  PRCP_SOL PRCP_SOL_new
cdo chname,QNS__ICE,A_Qns_ice       QNS__ICE QNS__ICE_new
cdo chname,QNS__OCE,A_Qns_oce       QNS__OCE QNS__OCE_new
cdo chname,QS___ICE,A_Qs_ice        QS___ICE QS___ICE_new
cdo chname,QS___OCE,A_Qs_oce        QS___OCE QS___OCE_new
cdo chname,RNF__OCE,A_Runoff        RNF__OCE RNF__OCE_new
cdo chname,TAUX_ICE,A_TauX_ice      TAUX_ICE TAUX_ICE_new
cdo chname,TAUX_OCE,A_TauX_oce      TAUX_OCE TAUX_OCE_new
cdo chname,TAUY_ICE,A_TauY_ice      TAUY_ICE TAUY_ICE_new
cdo chname,TAUY_OCE,A_TauY_oce      TAUY_OCE TAUY_OCE_new
cdo chname,QS___OCE,A_Qs_mix        QS___OCE QS___OCE_mix
cdo chname,QNS__OCE,A_Qns_mix       QNS__OCE QNS__OCE_mix

cdo merge CALV_OCE_new DQDT_ICE_new EVAP_ICE_new EVAP_TOT_new
PRCP_LIQ_new  PRCP_SOL_new  QNS__ICE_new  QNS__OCE_new  QS___ICE_new 
QS___OCE_new  RNF__OCE_new  TAUX_ICE_new  TAUX_OCE_new  TAUY_ICE_new 
TAUY_OCE_new  QS___OCE_mix QNS__OCE_mix rstas_orig.nc
\end{lstlisting}


\printglossaries