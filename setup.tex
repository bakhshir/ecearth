
\chapter{Setup}
%Compilation, running, make files, etc.
Short guides for Building and Running EC-Earth 3 on various machines is described at~\cite{br-wiki}.

Prior to compilation step, EC-EARTH build scripts need to be adjusted to the local environment. 
First, the source code of EC-Earth v3.2 HiRes development branch can be obtained from the svn repository: 
    
    \begin{center}
      \code{svn checkout \serifurl{http}{https://svn.ec-earth.org/ecearth3/branches/development/2016/r2811-hires}}
    \end{center}
    
Note that the main trunk is located at:

    \begin{center}
      \code{svn --username XX checkout \serifurl{http}{https://svn.ec-earth.org/ecearth3/trunk}}
    \end{center}


Secondly, make sure that \code{ExpandNodeList} and \code{makedepf90} are 
in the local \code{bin} folder. 
(\code{ExpandNodeList} is used to expand and collect node lists as used by SLURM, in order to perform set operations on them. 
The utility c\code{makedepf90} creates automatically Makefile dependency lists for Fortran source code; it is shipped together with EC-Earth, but the compiled version 
is a platform-dependent.)

The file \code{build-config.xml} in \code{sources} folder contains environmental variables and paths for the different platforms. 
KNMI works with the \emph{neuron} platform. The paths should be set properly, in particular, \code{ECEARTH\_SRC\_DIR} and libraries -- 
base directories of:
\begin{center}
\begin{tabular}{ll}
MPI & \code{MPI\_BASE\_DIR}\\ 
LAPACK & \code{LAPACK\_BASE\_DIR}\\ 
JPEG used for compression in HDF & \code{JPEG\_BASE\_DIR}\\ 
SZIP & \code{SZIP\_BASE\_DIR}\\ 
Hdf4 & \code{HDF4\_BASE\_DIR}\\ 
Hdf5 & \code{HDF5\_BASE\_DIR}\\ 
NetCDF & \code{NETCDF\_BASE\_DIR}\\ 
GRIB API & \code{GRIBAPI\_BASE\_DIR}\\ 
GRIBEX & \code{GRIBEX\_BASE\_DIR}\\ 
F90 dependency generator & \code{MAKEDEPF90}
\end{tabular}
\end{center}

For the development guide, see~\cite{dev-guide}.



\section{Compilation}

The compilation step requires:
\begin{itemize}
    \item Fortran 90 - fortran compiler
    ...
\end{itemize}



After setting up proper paths in \code{config-build.xml}, the compilation proceeds as follows:
\begin{verbatim}
    cconf32 & coas32 & cxios32 & crunoff32 & cifs32 & cnemo32  
\end{verbatim}
For more details on these functions, see Listing~\ref{app:comp} in the appendix. 
Note that XiOS is only compiled with g++ and its fortran interface is compiled with the available fortran compiler. As discussed below, it is not possible to use gfortran for the latter. Do not forget to run cconf32 if something is changed in config-build.xml file.

We have tested a local compilation of ec-earth 3.2b with the GNU compiler suite (gcc, g++ and gfortran) and the openMPI implementation of the message passing interface. In the build configuration file it is necessary to assign the following parameter values:
\begin{center}
\begin{tabular}{ll}
\texttt{FC}&\texttt{gfortran}\\
\texttt{FFLAGS}&\texttt{-O2 -g -fdefault-real-8 -fdefault-double-8 -ffree-line-length-none}\\
\texttt{FPP}&\texttt{gfortran -cpp}\\
\texttt{FFLAGS\_FREEFORM}&\texttt{-ffree-form}\\
\texttt{FFLAGS\_FIXEDFORM}&\texttt{-ffixed-form}\\
\texttt{FFLAGS\_FPP\_PREFIX}&\texttt{-D}\\
\texttt{MPI\_LIBS\_WITHOUT\_L}&\texttt{mpi mpi\_cxx mpi\_f77 mpi\_f90}\\
\end{tabular}
\end{center}
Unfortunately the Xios-1.0 component will not compile with the above settings in the current revision (2882). This is due to the fact that it uses token concatenation in its Fortran interface files, which is not substituted by gcc. Hence we advice strongly to install ifort before compiling ec-earth including Xios at this point. 

\section{Running}
There are three options of running EC-Earth on Bull with SLURM system~\cite{slurm}:
\begin{itemize}
  \item \code{sub32atm} for the atmosphere standalone model;    
  \item \code{sub32oce} for the ocean standalone model;    
  \item \code{sub32} for the coupled model.
\end{itemize}
These functions are presented in Listing~\ref{app:run} in the appendix. All functions have an option \code{-hr} for runs in high resolution. 